\documentclass{article}
\usepackage{fontspec}
% switch to a monospace font supporting more Unicode characters
\setmonofont{JuliaMono}
\usepackage{minted}
% instruct minted to use our local theorem.py
\newmintinline[lean]{lean4.py:Lean4Lexer -x}{bgcolor=white}
\newminted[leancode]{lean4.py:Lean4Lexer -x}{fontsize=\footnotesize}
%\usemintedstyle{tango}  % a nice, colorful theme
\usemintedstyle{xcode}

\begin{document}
\begin{leancode}
theorem funext {f₁ f₂ : ∀ (x : α), β x} (h : ∀ x, f₁ x = f₂ x) : f₁ = f₂ := by
  show extfunApp (Quotient.mk' f₁) = extfunApp (Quotient.mk' f₂)
  apply congrArg
  apply Quotient.sound
  exact h
\end{leancode}

\begin{leancode}
  def horn_filling_condition (X : SSet) (n i : Nat): Prop :=
    ∀ f : Λ[n, i] ⟶ X, ∃ g : Δ[n] ⟶ X,
    f = SSet.hornInclusion n i ≫ g
  
  /-- A simplicial set is called an ∞-category if it has the extension property
  for all inner horn inclusions `Λ[n, i] ⟶ Δ[n]`, n ≥ 2, 0 < i < n. -/
  def InfCategory := {X : SSet //
    ∀ (n i : Nat), n ≥ 2 ∧ 0 < i ∧ i < n → horn_filling_condition X n i}
\end{leancode}
\end{document}
